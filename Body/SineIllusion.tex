\graphicspath{{../Figure/sineIllusion/}{../Images/sineIllusion/}}

\renewcommand{\floatpagefraction}{.99}

\newcommand{\range}[1]{{\text{range}\left(#1\right)}}
\newcommand{\s}[2]{{_{#1}s^{ #2}}}
\newcommand{\atan}[1]{\text{atan}\left({#1}\right)}
\newcommand{\xR}{\mathbb{R}}

\newcommand{\done}[2][inline]{\todo[color=SpringGreen, #1]{#2}}  % for todos that have been seen and dealt with
\newcommand{\meh}[2][inline]{\todo[color=White, #1]{#2}}   % for todos that may no longer be relevant 
\newcommand{\comment}[2][inline]{\todo[color=SkyBlue, #1]{#2}} % for comments that may not be "to-do"s
\newcommand{\mcomment}[1]{\todo[color=SkyBlue]{#1}} % for margin comments

\newcommand{\newtext}[1]{\todo[inline, color=White]{ \color{OliveGreen}{#1}}} % new text - not necessarily something to be done
\newcommand{\newdo}[1]{\todo[inline, color=Plum]{#1}} % new to do item
\newcommand{\move}[1]{\todo[inline, color=Lime]{#1}} % new to do item


\begin{knitrout}
\definecolor{shadecolor}{rgb}{0.969, 0.969, 0.969}\color{fgcolor}\begin{kframe}


{\ttfamily\noindent\color{warningcolor}{\#\# Warning: cannot open file 'Code/sineIllusion/functions.r': No such file or directory}}

{\ttfamily\noindent\bfseries\color{errorcolor}{\#\# Error: cannot open the connection}}\end{kframe}
\end{knitrout}





\chapter{Signs of the Sine Illusion -- why we need to care}
%% Abstract
% Graphical representations have to be true to the data they display. Computational tools ensure this on a technical level. But we also need to take `flaws' of the human perceptual system into account. The sine illusion provides an example where human perception leads to systematic bias in the assessment of the optical stimulus, with a particularly notable impact on perception of time-series data with a seasonal component. In this paper, we discuss the reasons for the illusion and various strategies  useful to break the illusion or reduce its strength. We demonstrate the presence of the illusion in real-world and theoretical situations. We also present data from a user study which demonstrate the dramatic effect the sine illusion can have on conclusions drawn from displayed data.


\section{Introduction}
Graphics are powerful tools for summarizing large or complex data, but they rely on the main premise that any graphical representation of the data has to be ``true'' to the data \citep[see e.g.][]{tufte, wainer:2000, robbins:2005}. That is, a measurable quantity of a graphical element in the representation has to  directly reflect some aspect of the underlying data. Generally, we see a lot of discussion on keeping true to the data in the framework of (ab)using three dimensional effects in graphics. \citep{tufte} goes as far as defining a {\it lie-factor} of a chart as the ratio of the size of an effect in the data compared to the size of an effect shown, with the premise that any large deviations from one indicate a misuse of graphical techniques. Computational tools help us ensure technical trueness -- but this brings up the additional question of how we deal with situations that involve innate inability or trigger learned misperceptions in the audience. In this paper we want to raise awareness for one of these situations, show that it occurs frequently in our dealings with graphics and provide a set of strategies for solving or avoiding it.





























































