
\documentclass[11pt]{isuthesis}
\usepackage[]{graphicx}\usepackage[]{color}
%% maxwidth is the original width if it is less than linewidth
%% otherwise use linewidth (to make sure the graphics do not exceed the margin)
\makeatletter
\def\maxwidth{ %
  \ifdim\Gin@nat@width>\linewidth
    \linewidth
  \else
    \Gin@nat@width
  \fi
}
\makeatother

\definecolor{fgcolor}{rgb}{0.345, 0.345, 0.345}
\newcommand{\hlnum}[1]{\textcolor[rgb]{0.686,0.059,0.569}{#1}}%
\newcommand{\hlstr}[1]{\textcolor[rgb]{0.192,0.494,0.8}{#1}}%
\newcommand{\hlcom}[1]{\textcolor[rgb]{0.678,0.584,0.686}{\textit{#1}}}%
\newcommand{\hlopt}[1]{\textcolor[rgb]{0,0,0}{#1}}%
\newcommand{\hlstd}[1]{\textcolor[rgb]{0.345,0.345,0.345}{#1}}%
\newcommand{\hlkwa}[1]{\textcolor[rgb]{0.161,0.373,0.58}{\textbf{#1}}}%
\newcommand{\hlkwb}[1]{\textcolor[rgb]{0.69,0.353,0.396}{#1}}%
\newcommand{\hlkwc}[1]{\textcolor[rgb]{0.333,0.667,0.333}{#1}}%
\newcommand{\hlkwd}[1]{\textcolor[rgb]{0.737,0.353,0.396}{\textbf{#1}}}%

\usepackage{framed}
\makeatletter
\newenvironment{kframe}{%
 \def\at@end@of@kframe{}%
 \ifinner\ifhmode%
  \def\at@end@of@kframe{\end{minipage}}%
  \begin{minipage}{\columnwidth}%
 \fi\fi%
 \def\FrameCommand##1{\hskip\@totalleftmargin \hskip-\fboxsep
 \colorbox{shadecolor}{##1}\hskip-\fboxsep
     % There is no \\@totalrightmargin, so:
     \hskip-\linewidth \hskip-\@totalleftmargin \hskip\columnwidth}%
 \MakeFramed {\advance\hsize-\width
   \@totalleftmargin\z@ \linewidth\hsize
   \@setminipage}}%
 {\par\unskip\endMakeFramed%
 \at@end@of@kframe}
\makeatother

\definecolor{shadecolor}{rgb}{.97, .97, .97}
\definecolor{messagecolor}{rgb}{0, 0, 0}
\definecolor{warningcolor}{rgb}{1, 0, 1}
\definecolor{errorcolor}{rgb}{1, 0, 0}
\newenvironment{knitrout}{}{} % an empty environment to be redefined in TeX

\usepackage{alltt}
\newcommand{\SweaveOpts}[1]{}  % do not interfere with LaTeX
\newcommand{\SweaveInput}[1]{} % because they are not real TeX commands
\newcommand{\Sexpr}[1]{}       % will only be parsed by R


\usepackage{graphicx}
%---------------------------------------------------
\usepackage{color}
\usepackage[dvipsnames,svgnames]{xcolor}
\usepackage{wrapfig,float}
\usepackage{caption}
\usepackage{subcaption}
\usepackage{graphicx}
\usepackage{amssymb}
\usepackage{amsmath}
\usepackage{url}
\usepackage{ulem}
\usepackage[section]{placeins}
\usepackage{sidecap}
\usepackage{multirow}
\usepackage{bbm}
\usepackage[colorinlistoftodos]{todonotes}
%---------------------------------------------------

% Standard, old-style thesis
\usepackage{isutraditional}   \chaptertitle
% Old-style, thesis numbering down to subsubsection
\alternate
\usepackage{rotating}
% Bibliography without numbers or labels
\usepackage{natbib}
\bibliographystyle{apa}
%\includeonly{titletoc,chapter1}
%Optional Package to add PDF bookmarks and hypertext links
\usepackage[pdftex,hypertexnames=false,linktocpage=true]{hyperref}
\hypersetup{colorlinks=true,linkcolor=blue,anchorcolor=blue,citecolor=blue,filecolor=blue,urlcolor=blue,bookmarksnumbered=true,pdfview=FitB}

\newcommand{\done}[2][inline]{\todo[color=SpringGreen, #1]{#2}}  % for todos that have been seen and dealt with
\newcommand{\meh}[2][inline]{\todo[color=White, #1]{#2}}   % for todos that may no longer be relevant 
\newcommand{\comment}[2][inline]{\todo[color=SkyBlue, #1]{#2}} % for comments that may not be "to-do"s
\newcommand{\mcomment}[1]{\todo[color=SkyBlue]{#1}} % for margin comments

\newcommand{\newtext}[1]{\todo[inline, color=White]{ \color{OliveGreen}{#1}}} % new text - not necessarily something to be done

\newcommand{\move}[1]{\todo[inline, color=Lime]{#1}} % new to do item

%---------------------------------------------------




\begin{document}

\graphicspath{{Figure/LitReview/}{Images/LitReview/}}
\renewcommand{\floatpagefraction}{.99}





% Chapter 1 of the Thesis Template File
\chapter{OVERVIEW}

\section{Goals of Statistical Graphics}

\section{The Human Visual System}
Basic overview of structure, to serve as a reference

In order to design graphics for the human perceptual system, we must understand, at a basic level, the makeup of the perceptual system. There are multiple levels of perception that must correctly function in order to perceive visual stimuli successfully, but a somewhat simplistic higher-level analogy would be that we must understand both the hardware and software of the human visual system to create effective graphics.
The ``hardware", in this analogy, consists of the neurons that make up the eyes, optic nerve, and the brain itself. The higher-level functions (object recognition, working memory, etc.) comprise the ``software" component. In addition, much like computer software, there are different programs running simultaneously; these programs may interact with each other, run sequentially, or run in parallel. The following sections provide an overview of the grey-matter (hardware) components of the visual system as well as the higher-level cognitive heuristics (software) that order the raw input and construct our visual environment. 

\subsection{Hardware}
The physiology of perception is complex; what follows is a high-level overview of the physiology of perception, focusing on the areas most important to the perception of statistical graphics. This physiological information is important in understanding the difference between the sensation (i.e. the retinal image) and the perception (the corresponding mental representation), which is an important distinction in understanding how statistical graphics are perceived. 

\paragraph{The Eye}
The eye is a complex apparatus, but for our purposes, the primary component of the eye is the retina, which contains the sensory cells responsible for transforming light waves into electrical information in the form of neural signals. 
These sensory cells are specialized neurons, known as rods and cones, which perceive light intensity (brightness) and wavelength (color), respectively. One section of the retina, known as the fovea, contains only cones; the rest of the retina contains a mixture of rods and cones. Figure \ref{fig:retina} depicts the structure of the eye with a closeup of the retina. 

\begin{figure}
\centering
\includegraphics[width=.8\textwidth, keepaspectratio=TRUE]{Figure/LitReview/Retina}
\caption[The human eye, with closeup of receptor cells in the retina]{The human eye, with closeup of receptor cells in the retina (image from \protect\citealt{goldstein}, chap 3.1).} \label{fig:retina}
\end{figure}

\begin{figure}
\centering
\includegraphics[width=.8\textwidth, keepaspectratio=TRUE]{Figure/LitReview/AbsorptionSpectra}
\caption[Absorption spectra of retinal cells]{Absorption spectra of rods and short, medium, and long wave cones. (image from \protect\citealt{goldstein}, chap 3.3).} \label{fig:ColorRange}
\end{figure}

Figure \ref{fig:ColorRange} shows the responsiveness of rods and each of the three types of cones to wavelengths of light in the visual spectrum. This image suggests that we have relatively good visual discrimination of the yellow-green portion of the color spectrum, but relatively poor discrimination of colors in the red and blue portions of the color spectrum. As a result, rainbow-style color schemes are seldom appropriate for conveying numerical values, because the correspondance between the perceived information and the displayed information is not accurately maintained by the visual system \citep{rainbowcolor}. In addition, if any of the cones are missing or damaged as a result of genetic mutations, color perception is impaired, resulting in a smaller range of distinguishable colors. This set of impairments is known coloquially as color-blindness, and occurs in an estimated 5\% of the population. 

\paragraph{The Brain}
Once light hits the retina and causes a signal in the receptor cells, the information travels along the optic nerve and into the brain. Multiple neighboring rods are connected to the same neuron, where each cone is connected to its' own neuron. The combined wiring of rod cells is responsible for the Hermann grid illusion and the Mach bands seen in Figure \ref{fig:InhibitionIllusions}. Both of these illusions are a product of lateral inhibition, which is a result of the wiring of rod cells in the retina.  The specifics of the wiring of the receptor cells are somewhat complex; a more thorough explanation can be found in \citet{goldstein}, chapter 3.4. 

\begin{figure}
\centering
\begin{subfigure}[b]{.45\textwidth}
  \centering
  \includegraphics[width=\textwidth]{Figure/LitReview/HermannGrid}
  \caption{\small Hermann Grid Illusion \label{fig:hermanngrid}}
\end{subfigure}\hfill
\begin{subfigure}[b]{.45\textwidth}
  \centering
  \includegraphics[width=\textwidth]{Figure/LitReview/MachBands}
  \caption{\small Mach Bands. 
  \label{fig:machbands}}
\end{subfigure}
\caption[Inhibition Illusions]{Optical Illusions resulting from lateral inhibition. The Hermann Grid illusion causes dark circles to appear at the intersection of white lines; the Mach bands illusion causes the borders of adjacent rectangles to appear more strongly defined.} \label{fig:InhibitionIllusions}
\end{figure}

Once neural impulses have left the retina through the optic nerve, they travel to the visual cortex by way of several specialized structures within the brain that process lower-level signals. Receptor cells in the visual cortext respond to specific angles, spatial locations, colors, and intensities, and arrays of these special 'feature detector cells' process the information into a form that higher-level processes can utilize. These higher-level processes are what we have previously called 'software': they are not directly related to the physical brain, but they do seem to process information heuristically to produce higher-level reasoning and conclusions. In the next section, we explore some of the higher-level processes responsible for visual perception.

\subsection{Software}
Many of the processes for visual perception run simultaneously; in absence of a temporal ordering, we will start with the more basic tasks of visual perception and proceed towards higher-level processes. We will begin with attention.

\subsubsection{Attention and Perception} \label{AttentionPerception}
In many tasks, it is necessary to pay attention to many different input streams simultaneously; this is particularly true for complex tasks like driving a car. These tasks demand divided attention; the brain must process many different sources of information in parallel. By contrast, most image recognition tasks require selective attention, that is, focusing on specific objects and ignoring everything else. The brain accomplishes this attention through several mechanisms. 

Selective attention is accomplished by focusing the fovea (the area with the highest visual acuity) on the object. For instance, if the object is a page of text, each word will pass through the fovea, producing a focused stream of visual input. This stream of input consists of saccades (jumps between points of focus) and pauses in which the visual information is relayed to the brain. Figure \ref{fig:saccadestext} shows the saccades (lines) and pauses (circles) resulting when someone scans a paragraph of text. These saccades and pauses are utilized in eye-tracking technology to determine which parts of an image the observer is focusing on (and by extension, which information is being encoded by the brain). 

\begin{figure}[h]
\centering
\includegraphics[width=.5\textwidth]{SaccadesText}
\caption{A plot of saccades made while reading text. Saccades, shown by the lines, indicate ``jumps", while pauses are shown by circles, with size proportional to the time spent focusing on that area.}\label{fig:saccadestext}
\end{figure}

Selective attention is generally necessary for perception to occur, though there is some information that is encoded automatically. Experiments such as the fairly famous \href{http://www.theinvisiblegorilla.com/videos.html}{"gorilla" film}\footnote{\url{http://www.theinvisiblegorilla.com/videos.html}} demonstrate that even when there is attention focused on a task, information extraneous to that task is not always encoded, that is, even when participants focused on counting the number of passes of the basketball, they did not notice the obvious gorilla walking through the scene. It is important to understand which parts of a visual stimulus are the focus of a given perceptual task, because most of the information encoded by the brain is a result of selective attention. Eye-tracking can be an important tool useful to understand these perceptual processes, but participants are often able to report which parts of a stimulus contributed to their decision as well.

Within the brain, attention is important because it allows different regions of the brain which process color, shape, and position to integrate these perceptions into a multifaceted mental representation of the object \citep{goldstein}. This process, known as binding, is essential to coherently encode a scene into working memory. Feature integration theory \citep{treisman1980feature} suggests that these separate streams of information are initially encoded in the preattentive stage of object perception; focusing on the object triggers the binding of these separate streams into a single coherent stream of information. Many single features, such as color, length, and texture are \emph{preattentive}, because they can be pinpointed in an image without focused attention (and thus can be located faster), but specific combinations of color and shape require attention (because the features must be bound together) and are thus more difficult to search. Preattentive features are generally processed in parallel (that is, the entire scene is processed nearly simultaneously), while features requiring attention are processed serially. Examples of features processed  serially and in parallel are shown in Figure \ref{fig:parallelSerialFeatures}, taken from Chapter 6 of \citet{helander1997handbook}. The importance of preattentive processing to statistical graphics is discussed in Section \ref{LowLevelGraphics}. 

\begin{figure}[h]\centering
\includegraphics[width=.7\linewidth]{ParallelSerialFeatures}
\caption{Examples of features detected serially or in parallel (Chapter 6, \protect\citealt{helander1997handbook})}\label{fig:parallelSerialFeatures}
\end{figure}

Feature integration as a result of attention enables the brain to process a figure holistically. This processing is important for the most basic visual processes we take for granted, including object perception. 

\subsubsection{Object Perception} \label{ObjectPerception}
The most basic task of the visual system is to perceive objects in the world around us. This is an inherently difficult task, however, because the retina is a flat, two-dimensional surface responsible for conveying a three-dimensional visual scene. This dimensional reduction means that there are multiple three-dimensional stimuli that can produce the same visual image on the retina. This is known as the inverse projection problem - an infinite number of three-dimensional objects produce the same two-dimensional image. Less relevant to statistical graphics, but still complicating the object perception process, a single object can be viewed from a multitude of angles, in many different situations which may affect the retinal image (lighting, partial obstruction, etc). These problems mean that the brain must utilize many different heuristics to increase the accuracy of the perceived world relative to an ambiguous stimulus. 

The most commonly cited set of heuristics for object perception (and the set most relevant to statistical graphics) are known as the \emph{Gestalt Laws of Perceptual Organization} (\citealt{goldstein}, Chapter 5.2). These laws are related to the idea ``the whole is greater than the sum of the parts'', that is, that the components of a visual stimulus, when combined, create something that is more meaningful than the separate components considered individually. 

\begin{enumerate}
\item Pragnanz - the law of good figure. Every stimulus pattern is seen so that the resulting structure is as simple as possible. 
\item Proximity. Things that are close in space appear to be grouped. 
\item Similarity. Similar items appear to be grouped together. The law of similarity is usually subordinate to the law of proximity. 
\item Good Continuation. Points that can be connected to form straight lines or smooth curves seem to belong together, and lines seem to follow the smoothest path. 
\item Common Fate. Things moving in the same direction are part of a single group.
\item Familiarity. Things are more likely to form groups if the groups are familiar. 
\item Common Region. Things that are in the same region (container) appear to be grouped together
\item Uniform Connectedness. A connected region of objects is perceived as a single unit.
\item Synchrony. Events occurring at the same time will be perceived as belonging together.
\end{enumerate}

\begin{itemize}
\item Attention and perception

\item parallel processing - HCI handbook picture: summary of graphical perception

\item gestalt movement - wholistic approach

\item stored ``gist" and cognitive psych

\item Logarithmic perception (Sun \& Varshney) and Weber-Fechner. 
\end{itemize}

\subsection{Bugs}
\begin{itemize}
\item Optical Illusions - Accounting for perceptual oddities that are inherent in the visual system, how to avoid triggering those heuristics. Hermann grid, cafe wall, ponzo illusion
\item Colorblindness
\item 
\end{itemize}


\section{Statistical Graphics}
\subsection{Low-level perception of graphics}\label{LowLevelGraphics}
Healey's preattentive feature integration stuff
\subsection{Higher-level graphical perception}\label{HighLevelGraphics}
\begin{itemize}
\item Cleveland \& McGill
\item Heuristic guidelines i.e. Kosslyn and Tufte
\item Grammar of graphics allows us to test graphics which display the same information in different ways for comparable accuracy
\end{itemize}
Emphasize the low-level stuff vs. the higher-level stuff that focuses on preattentive information (i.e. not what we care about) and the gap in the middle where graphics are tested in more practical scenarios.

\subsection{Optical illusions and statistical graphics}
three-dimensional context, other ways to trigger optical illusions with graphics. Importance of being aware of the heuristics the brain uses so that problematic graphics can be avoided. 

\subsection{Interactive graphics?} 
Not sure if this should be a subsection, but graphics that respond to human interaction violate the two-dimensional graphic paradigm somewhat, which means there are a whole 'nother set of issues to consider... motion illusions, grouping, etc.

\section{Testing Statistical Graphics}
Methodology for testing graphical perception in humans. Talk about masking, the power of suggestion, habituation, etc. as well as paradigms - search and find, response time, numerical judgments, signal detection, eye tracking. 
\begin{itemize}
  \item Tukey
  \item Cleveland \& McGill
  \item Lineups
\end{itemize}

\end{document}
