\specialchapt{ABSTRACT}

There has been quite a bit of research on statistical graphics and visualization, generally focused on new types of graphics, new software to create graphics, interactivity, and usability studies. Our ability to interpret and use statistical graphics hinges on the interface between the graph itself and the brain that perceives and interprets it, and there is substantially less research on the interplay between graph, eye, brain, and mind than is sufficient to understand the nature of these relationships. 

The goal of the work presented here is to further explore the interplay between a static graph, the translation of that graph from paper to mental representation (the journey from eye to brain), and the mental processes that operate on that graph once it is transferred into memory (mind). Understanding the perception of statistical graphics should allow researchers to create more effective graphs which produce fewer distortions and viewer errors while reducing the cognitive load necessary to understand the information presented in the graph. 
% 
% Chapter \ref{litreview} contains a review of past literature which is relevant to the encoding and memory of statistical graphics, encompassing studies from psychology, psychophysics, meteorology, business, and statistics. The literature review encompasses many different areas of interdisciplinary research; to ensure that there is a reference for some of the more niche vocabulary in this dissertation, it begins with the physiology of the eye and includes a summary of some basic neuropsychology as well. From there, I discuss some of the psychology and psychophysics research that concerns visual perception, memory, and attention. This research roughly falls in the domain of `cognitive psychology', but includes studies from fields as diverse as meteorology and business. After establishing the underlying mechanisms of perception, I cover some of the research which focuses specifically on statistical graphics, and discuss some of the methodology used in these experiments.
% 
% 
% 
% Chapter \ref{sineillusion} discusses an optical illusion which is frequently present in even simple statistical graphics, but is incredibly difficult to resolve without altering the graph itself. This illusion (and its relative obscurity in the literature on graph perception) serves as a reminder that our graphs are only useful if they are designed with the perceptual system in mind. The presence of this illusion (and the solution for reducing the illusion's effects) serve as an introduction and case study for whether we can actually trust the things we see; the remainder of this dissertation aims to quantify variables which may affect how we answer the simple question ``What am I seeing?" in relation to statistical graphics. Chapter \ref{liefactor} expands on the discussion of the sine illusion, exploring the perceptual origin of the illusion and presenting a case study of an individual not affected by the illusion.
% 
% The use of statistical graphics is encouraged because graphics summarize statistical models and results in a form that is easy for most people to understand. The lineup protocol provides a convenient way to compare different types of graphics which display the same data, but results from comparative studies\citep{hofmann2012graphical} have demonstrated that individuals are highly variable in their ability to identify the target plot in a lineup. The study presented in Chapter \ref{visualreasoning} aims to explore the association between spatial reasoning, pattern recognition, and the ability to identify a target plot in a lineup successfully.
% 
% Finally, Chapter \ref{featurehierarchy} discusses a study which seeks to understand visually salient features of a plot, to aid researchers in creating graphics which are constructed to efficiently convey information visually. \citet{cleveland:1984} created a hierarchy of graphical tasks which have informed graph design for the past 30 years, ranking numerical estimations of graphical features by accuracy. Similarly, it would be useful to understand what features in a graph will dominate a viewer's perception: if linear trend information contradicts coloring of points, which feature will a viewer take away from the image? This experiment uses the lineup protocol to examine the features which are most salient to viewers, using pairwise comparisons of shape, color, linear trend, and outliers. 

Taken together, the experiments presented here should lay a foundation for exploring the perception of statistical graphics. There has been considerable research into the accuracy of numerical judgments viewers make from graphs, and these studies are useful, but it is more effective to understand how errors in these judgments occur so that the root cause of the error can be addressed directly. Understanding how visual reasoning relates to the ability to make judgments from graphs allows us to tailor graphics to particular target audiences. In addition, understanding the hierarchy of salient features in statistical graphics allows us to clearly communicate the important message from data or statistical models by constructing graphics which are designed specifically for the perceptual system. 
